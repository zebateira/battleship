\begin{vdm_al}
class CLI
 types
 values
 instance variables
 
 public static WATER  : int := 0;
 public static SHIP   : int := 1;
 public static HIT_SHIP  : int := 2;
 public static HIT_WATER : int := 3;
 
 public static markers : map int to char := {
  WATER  |-> ' ',
  SHIP  |-> 'O',
  HIT_SHIP |-> 'X',
  HIT_WATER |-> '+'
 };
 
 operations
 
  public static printInit : Player * Player ==> ()
   printInit (p1, p2) == (
   
    IO`println("\n********************************************");
    IO`println(  "*            Battleship War Game           *");
    IO`println(  "********************************************");
    
    IO`println(IO`freadval[VDMUtils`String]("res/header.txt").#2);
 
    IO`print(  "\n************* ");
    IO`print(p1.name);
    IO`print(" Vs ");
    IO`print(p2.name);
    IO`println(" **************\n");
    
   );
   
  public static printEnd : Player ==> ()
   printEnd (winner) == (
    IO`print("Player ");
    IO`print(winner.name);
    IO`println(" is Victorious! You fought with bravery young lad! ");
    IO`println("I shall award you with cookies and milk. Enjoy this feast to your content!\n");
    IO`println(IO`freadval[VDMUtils`String]("res/cookies_and_milk.txt").#2);
   );
 
  public static printHeaders : Player ==> ()
   printHeaders(player) == (
    IO`print("*** ");
    IO`print(player.name);
    IO`println("'s boards ***\n");
   );
 
  public static printBoard : Board ==> ()
   printBoard(board) == (
    dcl marker : int := WATER;
   
    IO`print("   ");
    for x = 1 to Board`BOARD_SIZE do (
     IO`print(x); 
     IO`print(" ");
    );
    IO`println(" ");
    
    for y = 1 to Board`BOARD_SIZE do (
     if y <> 10 then IO`print(" ");
     IO`print(y);
     IO`print("|");
     for x = 1 to Board`BOARD_SIZE do (
      let house in set board.houses be st house.x = x and house.y = y in (
      
       if house.hasShip then (
        if house.hit then
         marker := HIT_SHIP
        else
         (*@\notcovered{marker}@*) := (*@\notcovered{SHIP}@*);
       )
       else (
        if house.hit then (
         marker := HIT_WATER;
        )
        else (
         marker := WATER;
        )
       );

       IO`print(markers(marker));
       IO`print(" ");
      )
     );
     IO`print("|");
     IO`println(y);
    );
 
    IO`print("   ");
    for x = 1 to Board`BOARD_SIZE do (
     IO`print(x); 
     IO`print(" ");
    );
    IO`println(" ");
    IO`println(" ");
   );
  
 
  public static printBoardTabbed : Board ==> ()
   printBoardTabbed(board) == (
    dcl marker : int := WATER;
   
    IO`print("   ");
    IO`print("           ");
    for x = 1 to Board`BOARD_SIZE do (
     IO`print(x); 
     IO`print(" ");
    );
    IO`println(" ");
    
    for y = 1 to Board`BOARD_SIZE do (
     IO`print("           ");
     if y <> 10 then IO`print(" ");
     IO`print(y);
     IO`print("|");
     for x = 1 to Board`BOARD_SIZE do (
      let house in set board.houses be st house.x = x and house.y = y in (
      
       if house.hasShip then (
        if house.hit then
         marker := HIT_SHIP
        else
         marker := SHIP;
       )
       elseif house.hit then (*@\notcovered{marker}@*) := (*@\notcovered{HIT\_WATER}@*);

       IO`print(markers(marker));
       IO`print(" ");
       marker := WATER;
      )
     );
     IO`print("|");
     IO`println(y);
    );
 
    IO`print("   ");
    IO`print("           ");
    for x = 1 to Board`BOARD_SIZE do (
     IO`print(x); 
     IO`print(" ");
    );
    IO`println(" ");
    IO`println(" ");
   );
  
  public static printBoards : Player ==> ()
   printBoards(player) == (
    printHeaders(player);
    printBoard(player.boardplay);
    printBoardTabbed(player.boardown);
   );
  
end CLI
\end{vdm_al}
\bigskip
\begin{longtable}{|l|r|r|}
\hline
Function or operation & Coverage & Calls \\
\hline
\hline
printBoard & 97.4\% & 4 \\
\hline
printBoardTabbed & 97.6\% & 4 \\
\hline
printBoards & 100.0\% & 4 \\
\hline
printEnd & 100.0\% & 2 \\
\hline
printHeaders & 100.0\% & 4 \\
\hline
printInit & 100.0\% & 2 \\
\hline
\hline
CLI.vdmpp & 98.2\% & 20 \\
\hline
\end{longtable}

